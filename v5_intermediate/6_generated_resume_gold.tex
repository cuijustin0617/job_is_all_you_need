\documentclass[11pt, letterpaper]{article}

% Packages:
\usepackage[
    ignoreheadfoot,
    top=0.5cm,
    bottom=0.2cm,
    left=1.6cm,
    right=1.6cm,
    footskip=1.0cm,
]{geometry}
\usepackage{titlesec}
\usepackage{tabularx}
\usepackage{array}
\usepackage[dvipsnames]{xcolor}
\definecolor{primaryColor}{RGB}{0, 0, 0}
\definecolor{royalblue}{RGB}{65, 65, 185}
\usepackage{enumitem}
\usepackage{fontawesome5}
\usepackage{amsmath}
\usepackage[
    pdftitle={Justin Cui Resume},
    pdfauthor={Justin Cui},
    pdfcreator={LaTeX},
    colorlinks=true,
    urlcolor=primaryColor
]{hyperref}
\usepackage{eso-pic}
\usepackage{calc}
\usepackage{bookmark}
\usepackage{lastpage}
\usepackage{changepage}
\usepackage{paracol}
\usepackage{ifthen}
\usepackage{needspace}
\usepackage{iftex}

% Ensure that generated PDF is machine readable/ATS parsable:
\ifPDFTeX
    \input{glyphtounicode}
    \pdfgentounicode=1
    \usepackage[T1]{fontenc}
    \usepackage[utf8]{inputenc}
    \usepackage{palatino}
\fi

% Some settings:
\justifying
\AtBeginEnvironment{adjustwidth}{\partopsep0pt}
\pagestyle{empty}
\setcounter{secnumdepth}{0}
\setlength{\parindent}{0pt}
\setlength{\topskip}{0pt}
\setlength{\columnsep}{0.15cm}
\pagenumbering{gobble}

\titleformat{\section}{\needspace{4\baselineskip}\bfseries\large}{}{0pt}{}[\vspace{1pt}\titlerule]

\titlespacing{\section}{
    0pt
}{
    0.25 cm
}{
    0.15 cm
}

\renewcommand\labelitemi{\textendash}

\newenvironment{highlights}{
    \begin{itemize}[
        topsep=0.08 cm,
        parsep=0.08 cm,
        partopsep=0pt,
        itemsep=0pt,
        leftmargin=0.2 cm + 17pt
    ]
}
{
    \end{itemize}
}
\newenvironment{highlights2}{
    \begin{itemize}[
        topsep=0.08 cm,
        parsep=0.08 cm,
        partopsep=0pt,
        itemsep=0pt,
        leftmargin=0.2 cm + 10pt
    ]
}
{
    \end{itemize}
}

\newenvironment{subhighlights}{
    \begin{itemize}[
        topsep=0pt,
        parsep=0.05 cm,
        partopsep=0pt,
        itemsep=0pt,
        leftmargin=0.2 cm + 5pt
    ]
}{
    \end{itemize}
}

\newenvironment{onecolentry}{
    \begin{adjustwidth}{
        0 cm + 0.00001 cm
    }{
        0 cm + 0.00001 cm
    }
}{
    \end{adjustwidth}
}

\newenvironment{twocolentry}[2][]{
    \onecolentry
    \def\secondColumn{#2}
    \setcolumnwidth{\fill, 4.5 cm}
    \begin{paracol}{2}
}{
    \switchcolumn \raggedleft \secondColumn
    \end{paracol}
    \endonecolentry
}

\newcommand{\contactinfo}{
    \centering
    {\fontsize{25pt}{25pt}\selectfont Justin Cui} \\[4pt]
    \vspace{0.3cm}
    \small Toronto, ON, CA \hfill
    \href{mailto:justin.cui@mail.utoronto.ca}{justin.cui@mail.utoronto.ca} \hfill
    \href{tel:+1-416-827-9628}{+1 416 827 9628} \hfill
    \href{https://linkedin.com/in/Justin-Cui}{linkedin.com/in/Justin-Cui} \\
}

\begin{document}

\contactinfo
\vspace{2pt}

\section{Education}
\vspace{0.08 cm}
\begin{twocolentry}{09/2021 - 05/2026}
    {\textbf{University of Toronto}}, Bachelor of Applied Science
\end{twocolentry}
\vspace{0.11cm}
\begin{onecolentry}
\hspace{0.3cm}\textbf{Machine Intelligence},  cGPA: {\textbf{3.94}}/4.00
\vspace{0.07cm}
    \begin{highlights}
        \item Relevant Courses: Data Structures \& Algorithms, Probability and Statistics, Reinforcement Learning, Digital and Computer Systems, Introduction to Machine Learning, Matrix Algebra and Optimization, Probabilistic Reasoning, Software and Neural Networks, Introduction to Databases.
    \end{highlights}
\end{onecolentry}

\section{Publications}
\vspace{0.08 cm}
\begin{onecolentry}
    \begin{highlights}
    \item \textbf{Retrieval-Augmented Conversational Recommendation with Prompt-Based Semi-Structured Natural Language State Tracking} \\
    First-author, \textit{ACM SIGIR}, 2024.
    \vspace{0.1cm}
    \item \textbf{Elaborative Subtopic Query Reformulation for Query-Driven Recommendation} \\
    Co-author, \textit{ACM SIGIR}, 2025 (under review).
    \end{highlights}
\end{onecolentry}

\section{Experience}
\vspace{0.08 cm}
\begin{twocolentry}{May 2024 - Present}
    {\textbf{SWE/SDE/MLE/MLops Intern}}, Modiface -- Toronto, CA
\end{twocolentry}
\vspace{0.05 cm}
\begin{onecolentry}
    \begin{highlights}
        \item Engineered a robust data processing pipeline to augment a synthetic 3D-face dataset (25M images) using stable diffusion with Python and Shell scripting, achieving a 17\% improvement in facepoints prediction accuracy.
        \item Developed a seamless model conversion tool to translate in-house computer vision models from PyTorch to TensorFlow, ensuring consistent cross-platform integration, reducing latency by over 10\%.
        \item Optimized inference performance of generative AI models (including GANs and diffusion-based models) by integrating OpenVINO, reducing latency by over 40\% and enhancing scalability.
        \item Architected end-to-end fine-tuning pipeline for conversational AI systems, replacing in-context learning with supervised fine-tuning on 5k+ curated dialogue pairs.
        \item Refactored skin diagnosis repository using object-oriented design, consolidating recurring patterns into a unified sign registry and score normalization framework, enhancing code maintainability and consistency.
        \item Finetuned Stable Diffusion LoRA models to improve task-specific GenAI performance, improving accuracy by 40\%.
        \item Refined eyeliner extraction through advanced k-means clustering and adaptive HSV thresholding techniques, boosting accuracy in distinguishing facial makeup regions and optimizing the overall inference pipeline.
    \end{highlights}
\end{onecolentry}
\vspace{0.35 cm}

\begin{twocolentry}{April 2023 - September 2023}
    {\textbf{SWE/SDE/MLE/Applied ML/ML Research intern}}, Data-Driven Decision Making Lab (UofT) -- Toronto, CA
\end{twocolentry}
\vspace{0.1cm}
\begin{onecolentry}
    \begin{highlights}
        \item Designed and implemented a RAG-based chatbot system leveraging LLM APIs for dynamic natural language interactions and personalized responses.
        \item Architected the application using object-oriented design patterns, creating a modular system with 10+ interchangeable components for different NLP workflows.
        \item Developed core infrastructure in Python with 5+ API integrations, implementing rate limiting and caching mechanisms to handle 500+ RPM.
        \item Built a custom data pipeline using FAISS vector database to process and embed 1M+ product entries for real-time retrieval.
        \item Conducted comprehensive system testing including unit tests (95\% coverage) and integration tests with simulated synthetic users using LLMs.
        \item Delivered technical demonstrations to 20+ Meta engineers and executives, showcasing system architecture and RAG features.
    \end{highlights}
\end{onecolentry}
\vspace{0.35 cm}

\begin{twocolentry}{May 2022 - September 2022}
    {\textbf{Software Engineer Intern}}, Voith Hydro -- Montreal, CA
\end{twocolentry}
\vspace{0.1cm}
\begin{onecolentry}
    \begin{highlights}
        \item Engineered a SharePoint infrastructure solution to centralize engineering resources, deployed to 200+ users across the engineering department.
        \item Automated material specification analysis by creating Python scripts (Pandas/NumPy) to process CSV datasets, generating standardized reports that reduced manual review time.
        \item Conducted user research interviews with 15+ engineers to optimize UI/UX flow, resulting in 95\% adoption rate within first deployment month.
    \end{highlights}
\end{onecolentry}

\section{Projects}
\vspace{0.08 cm}
\begin{onecolentry}
    \begin{highlights}
    \item \textbf{Subjective Summarization with PCA + LLM} \\
    Developed a novel approach to summarize opinion-based datasets using LDA + LLMs to identify key topics and opposing viewpoints. Produced structured summaries that capture stance distributions and core arguments while maintaining interpretability.
    \vspace{0.1cm}
    \item \textbf{Geometric Approach to Query Performance Prediction} \\
    Explored a geometric approach to predicting query performance in information retrieval by analyzing embedding vector properties in high-dimensional space. Leveraged volume characteristics and spatial relationships to guide query reformulation and improve search effectiveness.
    \end{highlights}
\end{onecolentry}

\section{Skills}
\vspace{0.08 cm}
\begin{onecolentry}
    \begin{highlights}
        \item \textbf{Programming:} Python, C, MATLAB, PyTorch, TensorFlow, Keras, scikit-learn, Hugging Face Transformers, NumPy, Pandas, Matplotlib, Seaborn, OOP, Bash/Linux, Git, Docker, CI/CD, ML Ops, GCP, ONNX, OpenCV, Postgres, mySQL
        \item \textbf{Languages:} English (Native/Bilingual Proficiency), Mandarin (Native/Bilingual Proficiency), German (Limited Work Proficiency)
    \end{highlights}
\end{onecolentry}

\end{document}